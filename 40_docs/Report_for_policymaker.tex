\documentclass[12pt,letterpaper]{article}
\usepackage[utf8]{inputenc}
\usepackage[english]{babel}
\usepackage{amsmath}
\usepackage{amsfonts}
\usepackage{amssymb}
\usepackage{graphicx}
\usepackage{caption}
\usepackage{subcaption}
\usepackage[left=2cm,right=2cm,top=2cm,bottom=2cm]{geometry}
\author{White Team \\ Chenxi Rong, Xiaoquan Liu, Zhanyi Lin, Fabian Schmid}
\title{Estimate the Impact of Opioid Control Policies \\ (Report for Policy Maker)}
\begin{document}
\maketitle

\section{Motivation}
Opioids are a class of drugs that include prescription opioids (natural and semi-synthetic opioids and methadone), heroin, and synthetic opioids other than methadone (primarily fentanyl) that derive from, or mimic, natural substances found in the opium poppy plant and work in the brain to produce a variety of effects, including pain relief. Opioid drugs include prescription pain medicine and illegal drugs. Some people may experience euphoria, a joyful sensation of well-being, from opioids, whether they are legally prescribed or not. Opioids do not always generate euphoria, but for those who do, there is a chance they will be used again and again because of how they feel. Therefore, even with a doctor's supervision, using opioids can pose risks. A person's tolerance and dependency to prescription drugs can develop over time, necessitating greater and more frequent dosages, finally leading to addiction and the person's turning to illegal markets in order to maintain their addiction, and subsequently causing death.

According to the Centers for Disease Control and Prevention (CDC), the number of drug overdose deaths has quintupled since 1999, and the rise in opioid overdose deaths can be outlined in three distinct waves: the first wave in the 1990s with increased prescribing of opioids, the second wave in 2010 with heroin, and the third wave in 2013 with synthetic opioids like fentanyl. In order to fight the opioid overdose epidemic, policymakers have made policy interventions to limit the over prescription of opioids. Texas regulations with regard to treating pain with controlled substances went into effect in January 2007. Florida’s legislature became effective in 2010, and a series of changes related to drug prescription took place in the following years. Washington regulated the prescribing requirements of opioids for pain treatment in January 2012, which included periodic patient reviews, milligram thresholds, strict documentation guidelines, and consultations with pain management experts.

For all three of these policy changes, we performed both a pre-post analysis and a difference-in-difference analysis to understand the effect of opioid drug regulations on both the amount of opioid shipments and drug overdose deaths. For the pre-post analysis, we will demonstrate the trend of overdose deaths and opioid shipments over years. After the policy went into effect, we analyze if our plots show a difference in the trend in each state right before the policy became effective and right it. To further valid our analysis, we conduct a difference-in-difference analysis. We look at the development in the state with a policy change over time and compare it to similar states which are not affected by the policy.

We used drug overdose death data from the US Vital Statistics records, prescription opioid drug shipments from the Washington Post, and US census population data. The analysis was based on a large number of observations and credible data sources. Nevertheless, the pre-post and the diff-in-diff analysis both rely on crucial assumptions to evaluate the policies sufficiently. For the pre-post analysis, we have to assume that no other event took place at the same time of the policy implementation which might also effect the overdose deaths or the shipped opioids. This might be violated if the US Customs Service managed to dramatically reduce the importation of fentanyl into the United States at the same time Florida’s policy went into effect. This would likely reduce the number of overdose deaths throughout the United States. The diff-in-diff analysis is more flexible but still requires to assume that the state with a policy change and the states without a policy change behave similarly if no policy would have been implemented. These assumptions has to be taken into account, when reading through the following analysis. 

\section{Analysis}
\subsection{Florida}

\textbf{Opioid shipment}

\begin{figure}[!h]
\centering
\begin{subfigure}{.5\textwidth}
  \centering
  \includegraphics[width=.9\linewidth]{../30_results/General_Results/florida_opioid_shipment_prepost.png}
  \caption{Pre-post analysis}
  \label{fig:fl_ship_prepost}
\end{subfigure}%
\begin{subfigure}{.5\textwidth}
  \centering
  \includegraphics[width=.9\linewidth]{../30_results/General_Results/florida_opioid_shipment_diffdiff.png}
  \caption{Diff-in-diff analysis}
  \label{fig:fl_ship_did}
\end{subfigure}
\caption{Analysis of opioid policy on opioid shipments in Florida}
\label{fig:fl_ship}
\end{figure}

In the left graph in Figure \ref{fig:fl_ship}, we can see an increasing trend in opioid shipments per capita which reverses after the policy is introduced. In the right graph, we compared the development of opioid shipments in Florida with the one in Georgia, North Carolina, and South Carolina since those three states are close to Florida. The output shows that in the states without a policy change, the number of shipped opioid per capita is still increasing over time. that the slope of control groups is still positive but the slope of Florida opioid shipment converts to negative. Therefore, we may conclude that the decrease of the opioid per capita in Florida after 2010 is caused by the policy, which implies that the policy is effective. \\

\noindent \textbf{Overdose death} \\

\begin{figure}[!h]
\centering
\begin{subfigure}{.5\textwidth}
  \centering
  \includegraphics[width=0.7\linewidth]{../30_results/General_Results/florida_overdose_death_prepost.png}
  \caption{Pre-post analysis}
  \label{fig:fl_death_prepost}
\end{subfigure}%
\begin{subfigure}{.55\textwidth}
  \centering
  \includegraphics[width=1\linewidth]{../30_results/General_Results/florida_overdose_death_diffdiff.png}
  \caption{Diff-in-diff analysis}
  \label{fig:fl_death_did}
\end{subfigure}
\caption{Analysis of opioid policy on overdose deaths in Florida}
\label{fig:fl_death}
\end{figure}


From the output of the pre-post graph in Figure \ref{fig:fl_death}, we can observe that the number of overdose deaths in Florida were rising before the policy implementation. However, in the years after the policy went into effect, there is a negative trend visible in the plot. Looking at the difference-in-difference analysis, we can see that in the selected states without a policy change the number of overdose deaths per 100,000 inhabitants is still growing.  Therefore, we may conclude that the decrease of the overdose deaths per capita in Florida after 2010 is because of the policy, which means the policy is effective. Therefore, we infer that the policy is effective in reducing overdose deaths in Florida.

\subsection{Texas}
\textbf{Opioid shipment}

\begin{figure}[!h]
\centering
\begin{subfigure}{.5\textwidth}
  \centering
  \includegraphics[width=0.7\linewidth]{../30_results/Bonus_Results/tx_monthly_prepost_successful.png}
  \caption{Pre-post analysis}
  \label{fig:tx_ship_prepost}
\end{subfigure}%
\begin{subfigure}{.55\textwidth}
  \centering
  \includegraphics[width=0.7\linewidth]{../30_results/Bonus_Results/tx_monthly_did_notsure.png}
  \caption{Diff-in-diff analysis}
  \label{fig:tx_ship_did}
\end{subfigure}
\caption{Analysis of opioid policy on opioid shipments in Texas}
\label{fig:tx_ship}
\end{figure}
Since the opioid shipment data is only available from 2006 on, meaning we only have one year of data before the policy change, we analyzed opioid shipment data by month instead of by year. The pre-post chart in Figure \ref{fig:tx_ship} shows a increasing trend in opioid shipments in Texas before and after the policy went into effect. However, increase is less steep after the policy implementation in January 2007. Therefore, we may conclude that the policy restricted the opioid shipment amount by a small amount. For the difference-in-difference analysis, we selected Arkansas, Oklahoma, and New Mexico as control groups since those three states are close to Texas. After the policy implementation, the trend in Texas and the control states is similar. Thus, we infer that the opioid policy in Texas was not successful in controlling the opioid shipments. \\

\noindent \textbf{Overdose death}

\begin{figure}[!h]
\centering
\begin{subfigure}{.5\textwidth}
  \centering
  \includegraphics[width=0.7\linewidth]{../30_results/General_Results/texas_overdose_death_prepost.png}
  \caption{Pre-post analysis}
  \label{fig:tx_death_prepost}
\end{subfigure}%
\begin{subfigure}{.55\textwidth}
  \centering
  \includegraphics[width=1\linewidth]{../30_results/General_Results/texas_overdose_death_diffdiff.png}
  \caption{Diff-in-diff analysis}
  \label{fig:tx_death_did}
\end{subfigure}
\caption{Analysis of opioid policy on overdose deaths in Texas}
\label{fig:tx_death}
\end{figure}

From the output of the pre-post graph, we can tell that the overdose death per 100,000 people increased year by year before the policy change and started to decrease annually after the policy's effective date. The difference-in-difference analysis reveals that neighboring states without a policy change witness an increase in overdose deaths while Texas has a negative trend after the policy implementation. Therefore, we may conclude that the decrease in overdose deaths per capita in Texas after 2007 is caused by the policy, which means that the policy is effective.

\subsection{Washington}
\textbf{Opioid shipment}

\begin{figure}[!h]
\centering
\begin{subfigure}{.5\textwidth}
  \centering
  \includegraphics[width=0.7\linewidth]{../30_results/General_Results/washington_opioid_shipment_prepost.png}
  \caption{Pre-post analysis}
  \label{fig:wa_ship_prepost}
\end{subfigure}%
\begin{subfigure}{.55\textwidth}
  \centering
  \includegraphics[width=1\linewidth]{../30_results/General_Results/washington_opioid_shipment_diffdiff.png}
  \caption{Diff-in-diff analysis}
  \label{fig:wa_ship_did}
\end{subfigure}
\caption{Analysis of opioid policy on opioid shipments in Washington}
\label{fig:wa_ship}
\end{figure}

The pre-post analysis in Figure \ref{wa_ship} indicates that there is a positive trend in opioid shipments in Washington before the policy is implemented. This trend changes after the policy went into effect in 2012 and is since then slightly negative. Even though the fall in opioid shipments is not very pronounced, the overall trend is totally different from the previous years. For the difference-in-difference analysis, we selected Oregon, Idaho, and Montana as control groups since those three states are close to Texas. From the output, we can see that the development after the policy implementation is similar for Washington and the states in the control group. Hence, the pre-post analysis and the diff-in-diff analysis provide opposing conclusion. Due to the statistical superiority of the latter, we cannot conclude that the policy is successful in Washington with regards to shipped opioids. \\

\noindent \textbf{Overdose death}

\begin{figure}[!h]
\centering
\begin{subfigure}{.5\textwidth}
  \centering
  \includegraphics[width=0.7\linewidth]{../30_results/General_Results/washington_overdose_death_prepost.png}
  \caption{Pre-post analysis}
  \label{fig:wa_death_prepost}
\end{subfigure}%
\begin{subfigure}{.55\textwidth}
  \centering
  \includegraphics[width=1\linewidth]{../30_results/General_Results/washington_overdose_death_diffdiff.png}
  \caption{Diff-in-diff analysis}
  \label{fig:wa_death_did}
\end{subfigure}
\caption{Analysis of opioid policy on overdose deaths in Washington}
\label{fig:wa_death}
\end{figure}

From the pre-post graph in Figure \ref{wa_death}, we can tell that number of overdose deaths per 100,000 people in Washington was both increasing before and after the policy became effective. Though briefly after the policy implementation, the overdose deaths per capita became lower, they still increased year by year after the policy's effective date by an even larger number. Combining this insight with the output from the difference-in-difference analysis, we can see that the trends for both Washington and states in the control group are the same before and after the policy became effective. Therefore, we can conclude that the policy was not successful in controlling the increasing trend of overdose deaths in Washington.


\section{Conclusion}

Based on the graphs of our pre-post and difference-in-difference analysis, Florida's drug policy was effective in decreasing the prescription opioid shipment amount as well as decreasing the overdose deaths. Texas' drug policy was not successful in controlling the opioid shipments but did decrease the overdose deaths. However, Washington's drug policy was not successful in both controlling the opioid shipments and decreasing the overdose deaths.

By comparing policies in the three states, we can learn that since there were a series of changes in Florida that strictly prohibited physician dispensing of drugs and penalized it by arrest, seizure of assets, and clinic closure, strict policies combined with the practical implementation of penalties achieve that the policy have the desired effect. However, policies in both Texas and Washington were mainly focused on performing patient evaluation and consulting, obtaining patients' consent, and making dose recommendations. Those policies seem to be too moderate to be effective, and in the meantime, no severe penalty was introduced. Furthermore, policies in Florida focused on cutting off the source of prescription opioids, which reduced opioid dispensing to the greatest extent possible. To sum up, the problem's root cause should be considered when making policies, and strict rules that are well-applied can assist in producing the intended effects.


\end{document}
